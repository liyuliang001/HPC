\documentclass[a4paper]{report}   % list options between brackets
\usepackage{CJK}
\usepackage{fullpage}
\usepackage{graphicx}
\usepackage{tabularx}
\usepackage{listings}
\usepackage{rotating}
\usepackage{amsmath}
\usepackage{cite}

% type user-defined commands here

\begin{document}
\begin{CJK}{UTF8}{gbsn}
\title{\vspace{120pt}\huge High Performance Computing\\ Homework 1\\[1em] \huge Understanding the Programming Platform\\[1em] \Large Report}
\author{\\[3em]\Large Peiyun Hu, 2010011297 \\ [1em]\Large Department of Computer Science \& Technology}
\date{Sep. 22, 2012}    % type date between braces
\maketitle

\chapter{Environment}             % chapter 1
\section{Hardware}
\subsection{CPU} \label{info:cpu}

Model Name: Intel(R) Core(TM)2 Quad CPU @ 2.93GHz\\
Cores: 4
\subsection{Memory} 
Total Memory: 4055940 KB

\subsection{Cache}
Cache Size: 4096 KB

\subsection{Disk} 


\section{Software}     % section 1.1

\subsection{OS} 
Linux version 2.6.38-8-generic
\subsection{Compiler} 
gcc version 4.5.2 (Ubuntu/Linaro 4.5.2-8ubuntu4)
\subsection{Compiler Options}

\subsection{Timing Method} 


\chapter{Performance}
\section{Theoretical Peak Performance}
As is shown in \ref{info:cpu}, it is known that A Core 2 Quad has 4 cores, and 1 execution unit per core. A SSE register is 128 bits wide, and can store 4 floats per register, for a float is 32 bit wide. Assuming that the execution unit does 1 SSE operation per cycle, the theoratical peak performance should be like:\

\begin{center} 
4 Cores * 1 Execution Unit * 2.93 GHz * 1 SSE Operation Per Second * 4 Float Operations, 
\end{center}

\noindent and the result is as below:
\begin{align}
\text{Peak Performance of Flops = 46.88 GFlops}
\end{align}

Similarly, we could work out the theoretical peak performance of Double Operations per second, which is half of Peak Performance of Flops. \cite{plain:myarticle}

\section{Experiment Results}
\subsection{Vector-Vector}

\subsection{Matrix-Vector}

\subsection{Matrix-Matrix}


%\begin{thebibliography}{9}
%  % type bibliography here
%  Peiyun, how to make love, 2012, 1
%\end{thebibliography}

\bibliographystyle{plain}
\bibliography{bibitex}

\end{CJK}
\end{document}